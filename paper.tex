\documentclass[twocolumn,jsaiac]{jarticle}
\bibliographystyle{jsai}
%!TEX encoding = UTF-8 Unicode

\usepackage{multirow}
\usepackage{jsaiac}
\usepackage{bm}
\usepackage[dvipdfmx]{graphicx}
\usepackage[hyphens]{url}

\title{
    \jtitle{テクニカル分析の有効性を検証する}
}

\author{
    \jname{37-147308 飯塚 修平}
}

\affiliate{
  \jname{東京大学大学院 工学系研究科 技術経営戦略学専攻}
}

\def\Style{``jsaiac.sty''}
\def\BibTeX{{\rm B\kern-.05em{\sc i\kern-.025em b}\kern-.08em%
 T\kern-.1667em\lower.7ex\hbox{E}\kern-.125emX}}
\def\JBibTeX{\leavevmode\lower .6ex\hbox{J}\kern-0.15em\BibTeX}
\def\LaTeXe{\LaTeX\kern.15em2$_{\textstyle\varepsilon}$}

\newcommand{\argmax}{\mathop{\rm arg~max}\limits}
\begin{document}
\maketitle
\section{はじめに}
近年、インターネット証券の普及によって個人投資家も株式証券の売買が容易にできるようになった。
国内でも NISA と呼ばれる少額投資非課税制度が導入され、中長期投資促進への期待が寄せられている\cite{nisa}。
また、2007 年より東京証券取引所が株式証券の売買単位を 100 株に統一するよう指導しており、
単位が引き下げられたことによって少額から株式証券を購入することができる環境が整ってきている\footnote{
  売買単位の集約に向けた行動計画 - 日本取引所グループ \url{http://www.jpx.co.jp/equities/improvements/unit/tvdivq00000050ft-att/keikaku.pdf}}。

投資家は運用成績を向上させるために株式の分析を行うが、その方法は大きく{\bf ファンダメンタル分析}と{\bf テクニカル分析}のふたつに分けることができる。
ファンダメンタル分析では、理論株価に基づいて株価を予測して投資判断を行う。
財務諸表から計算される財務指標に加えて、ニュースや IR 情報などの情報をもとに株価を予測する定性的分析手法である。
一方、テクニカル分析では、移動平均線に基づいて投資判断を行う。
過去の時系列データをもとに株価の値動きを予測する定量的分析手法である\cite{ga}。

今後、ますます個人投資家の株取引が容易になると、
日中に別の仕事を持つ個人投資家の数も増えてくるものと考えられる。
しかし彼らは専業トレーダーとは異なり、時間的な制約によって様々な銘柄の株価を常時チェックすることは難しいため、ある程度自動化された売買システムが必要になる。
そこで近年注目を浴びているのが、一定の売買ルールにしたがって取引を行う{\bf システムトレード}と呼ばれる手法である\cite{short}。
システムトレードはアルゴリズムに基づいた取引を行うため、定性的なファンダメンタル分析を取引に組み込むことは難しい。
一方、テクニカル分析は定量データに基づいて行われるため、アルゴリズムとして記述することさえできればシステムトレードに組み込むことが可能になる。
したがって、今後ますますテクニカル分析の重要性が増してくるものと考えられる。

そこで本稿では、テクニカル分析手法の中でもシンプルで広く用いられている{\bf グランビルの法則}にしたがって株式証券の取引を行うエージェントを作成してシミュレーションを行い、
テクニカル分析が運用成績向上にもたらす効果を検証する。
直近 1 年間に国内で取引された上場株の値動きデータを用いて評価実験を行った結果、
ランダムに取引を行う場合と比べてテクニカル分析に基づいて取引を行うエージェントは有意に良い運用成績を収めることができなかった。
しかし、テクニカル分析は売買する対象の銘柄を絞り込む効果があることに加え、さらに多面的に値動きを見る機能を加える事によって、運用成績を向上することができる可能性を示すことができた。
本稿で示した評価実験の結果および考察によって、システムトレードを活用する個人投資家の運用成績向上に貢献できるものと考えている。

\section{関連研究}
定量的な値動きデータを対象にするテクニカル分析を元にして、高い運用成績を収めるアルゴリズムを開発する研究はいくつか行われている。
杉本らは株価の時系列データに決定木学習を用いて、未来状態予測を行うアルゴリズムを開発している\cite{decisiontree}。
他にも遺伝的アルゴリズムや遺伝的プログラミングなどのメタヒューリスティクスを用いた学習を行った手法もいくつか提案されており\cite{ga, gp}、
値動きデータに基づく株価予測はデータマイニングの分野でよく取り上げられる研究テーマである。

一方、ファンダメンタル分析にもデータマイニングを適用して自動化しようとする試みは行われている。
和泉らは、国際金融情報センター\footnote{公益財団法人 国際金融情報センター\url{http://www.jcif.or.jp/}}が公開する市場解説記事にテキストマイニングを用いることで、
取引判断を行うエージェントを作成した\cite{multiagent}。
ファンダメンタル分析を行うエージェントとテクニカル分析を行うエージェントが混合した人工市場におけるシミュレーション実験も行われている\cite{short}。

このように、データマイニングによって得られた知識をもとに、自動で高い運用成績を目指したエージェントを開発する試みは行われている。
しかし、その多くは時系列データに運用成績の向上のために機械学習手法を適用したものであり、機械学習について詳しくない個人が投資判断材料として
活用しやすいものではない。

\section{グランビルの法則}
それに対して、株価データの目に見えるパターンに着目して取引タイミングを分析する手法としてグランビルの法則がある。
グランビルの法則は、過去の株価データの平均移動線を利用したものであり、日々の株価の推移と見比べることによって、売買タイミングを決定する\cite{granville}。

グランビルの法則では、株式証券を買うタイミングとして、{\bf ゴールデンクロス}を提案している。
ゴールデンクロスとは、短期の移動平均線が長期の移動平均線を下から上に突き抜けた点のことであり、
株価上昇の勢いが強いことを示唆するとされている。逆に、短期の移動平均線が長期の移動平均線を上から下に切った場合は、
株価の勢いが弱い{\bf デッドクロス}とされている\cite{goldencross}。

この定義はシンプルでわかりやすく、株価チャートを見るだけで当該の場所を発見しやすい。
そのため、データマイニングや時系列データ分析について知見が無い個人投資家にも使いやすいツールとして親しまれている。
こういった判断基準が明確な分析手法のほうが、投資の成否の原因がはっきりするため、投資というリスクがつきまとう活動に対する手法には向いていると考えられる。
そこで本稿では、このグランビルの法則、特にゴールデンクロスに着目して投資判断をした場合に、高い運用成績を収めることができることを検証する。

\section{提案手法}
\section{評価実験}
\section{考察}
\section{まとめ}

\bibliography{references}
\bibliographystyle{jsai}

\end{document}
